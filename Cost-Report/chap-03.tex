\chapter{Applying COCOMO}
COCOMO is one of the most generally used software estimation models in the world. COCOMO predicts the efforts and schedule of a software product based on the size of the software.

Here, we have applied COCOMO to estimate the costs of our project.

\section{Software Function Points}
A Function Point (FP) is a unit of measurement to express the amount of business functionality, an information system (as a product) provides to a user. FPs measure software size. They are widely accepted as an industry standard for functional sizing.

Our project contains following function points:

\begin{enumerate}
 \item App Preference Management
 \item SIP Registration
 \item Initiate Outgoing Call
 \item Receive Incoming Call
 \item Hangup a Call
 \item Encode / Decode Voice Data
 \item Record Voice from Microphone
 \item Send Voice over the network
 \item Receive Voice over the network
 \item Play Voice to Earpiece / Speaker
 \item Manage Loudspeaker Mode
 \item Perform Call Hold / Unhold
 \item Do Mute / Unmute Microphone
 \item Generate following UI Pages:
 \begin{itemize}
  \item App Splash Screen
  \item Home Page
  \item Settings Page
  \item Incoming Call Screen
  \item Active Call Screen
 \end{itemize}

\end{enumerate}

\section{Lines of Code}
Source lines of code (SLOC), also known as lines of code (LOC), is a software metric used to measure the size of a computer program by counting the number of lines in the text of the program's source code.

Our project is estimated to contain approximately \textbf{2,000} lines of code ($2.0 KLOC$). It excludes auto-generated Java source code and JNI library implementation of entire PJSIP, PJSUA and PJMEDIA stack along with different audio codecs and channel management.

\section{Cost Calculation}
The basic COCOMO equation takes the form:
$$Effort = a_1 \times (KLOC) ^ {a_2}$$
$$Tdev = b_1 \times (efforts) ^ {b_2}$$

Our project size is estimated to be \textbf{2.0 KLOC}.

This is a service oriented, organic category of project. Here, $a_1 = 2.4$, $a_2 = 1.05$, $b_1 = 2.5$, $b_2 = 0.38$. \\
Therefore, \\
\\
Effort, $E = 2.4 \times (2.0) ^ {1.05} = 4.97$ PM \\
Tdev, $D = 2.5 \times (4.97) ^ {0.38} = 4.60$ PM.
\\
\\
Monthly average salary of each developer involved in this project team is around \textbf{30,000 TK}. \\
Cost for employee salaries = $30,000 \times 4.60 = 1,38,000$ TK.\\
Company revenue policy is \textbf{30\%}.

Therefore, total costs calculated is: $1,38,000 \div ({1 - 0.30}) = 1,97,143$ TK.

\section{Summary}
Using an estimation of \textbf{2,000} lines of code with employees of an average of \textbf{30,000} TK monthly salaries, the project cost is given bellow including \textbf{30\%} company's revenue:
\\
\begin{center}
    \huge 1,97,143 TK.
\end{center}
